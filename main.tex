% -*- coding: UTF-8; -*-
%ne pas modifier ou enlever la ligne ci-haut
\documentclass[11pt,letterpaper]{article}

%2015-08-26 - Document préparé par David Lafrenière, pour le cours PHY1234.

%Pour langue et caractères spéciaux
\usepackage[french]{babel} 
\usepackage[T1]{fontenc}
\usepackage{lmodern}
\usepackage[utf8]{inputenc}
\usepackage{multicol}
\usepackage{gensymb}
\usepackage{MnSymbol}
\usepackage{xcolor}
\usepackage{soul}
\usepackage{units}
\usepackage{makecell}
\usepackage{wasysym}
\usepackage{subfig}
\usepackage{listings}
\usepackage{amsmath}
\usepackage{physics}

\definecolor{codegreen}{rgb}{0,0.6,0}
\definecolor{codegray}{rgb}{0.5,0.5,0.5}
\definecolor{codepurple}{rgb}{0.58,0,0.82}
\definecolor{backcolour}{rgb}{0.95,0.95,0.92}
 
\lstdefinestyle{mystyle}{
    backgroundcolor=\color{backcolour},   
    commentstyle=\color{codegreen},
    keywordstyle=\color{magenta},
    numberstyle=\tiny\color{codegray},
    stringstyle=\color{codepurple},
    basicstyle=\ttfamily\footnotesize,
    breakatwhitespace=false,         
    breaklines=true,                 
    captionpos=b,                    
    keepspaces=true,                 
    numbers=left,                    
    numbersep=3pt,                  
    showspaces=false,                
    showstringspaces=false,
    showtabs=false,                  
    tabsize=2
}
 
\lstset{style=mystyle}

%Pour ajuster les marges
\usepackage[top=2cm, bottom=2cm, left=1.4cm, right=1.4cm, columnsep=20pt]{geometry}


%pour inclure des graphiques
\usepackage{graphicx}
\usepackage{float}
\usepackage{amsmath}

%Pour inclure des adresse web
\usepackage{url}

%pour inclure les codes en annexe
\usepackage{fancyvrb}
\usepackage{listingsutf8}
\usepackage{color}
\lstset{inputencoding=utf8/latin1,numbers=left,numberstyle=\footnotesize,frame=single,commentstyle=\it\color{blue},keywordstyle=\bf\color{red}}

\begin{document}

%Page titre
\begin{titlepage}
\center

\vspace*{2cm}

\textsc{\LARGE Université de Montréal}\\[1cm] 
\textsc{\Large }\\[1.5cm] 

\rule{\linewidth}{0.5mm} \\[0.5cm]
{\LARGE \bfseries Project 1} \\[0.2cm] % ***éditez ceci***
\rule{\linewidth}{0.5mm} \\[5cm]

\textbf{\Large  Martaillé Richard Eloi}\\

{\Large Matricule : 20133914}\\[8cm] 

\end{titlepage}




\section{Problem 1}

We have the function $u(x)$ define as:

\begin{equation}
    u(x) = 1 - (1 - e^{-10})x - e^{-10x}
\end{equation}

In order to check if equation (1) is a solution to the ODE, we calculate $-\frac{d^2u}{dx^2}$ to see if we found the function $f(x)=100e^{-10x}$.

\begin{align*}
    \frac{du}{dx} &= 1 - e^{-10} +10 e^{-10x} \\
    \Longrightarrow -\frac{d^2u}{dx^2} &= -100e^{-10x} = f(x)
\end{align*}

We have found that the function u defined is indeed a solution to our ODE and the last verification we need is if $u(x)$ respect our boundaries conditions. \\

We can see for $x=0$, we have $u(0)= 1 - 0 -1 =0$ and for $x=1$ we have $u(1) = 1 - 1 + e^{-10} - e^{-10} = 0$. Our function $u(x)$ respect both boundaries conditions and the second derivative is equal to $-f(x)$, hence we have shown that $u(x)$ is a solution to our ODE.



\section{Problem 2}

\section{Problem 3}

\section{Problem 4}

\section{Problem 5}

\section{Problem 6}

\section{Problem 7}

\section{Problem 8}

\section{Problem 9}

\section{Problem 10}





\end{document}